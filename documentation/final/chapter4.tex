\chapter{Technologie użyte do zrealizowania projektu}
Istotnym elementem pracy związanej z powstaniem gry było poznanie nieznanych mi 
wcześniej technologii, które nie były używane w ramach zajęć na studiach 
inżynierskich. W tym rozdziale chciałbym przedstawić te technologie oraz 
wyjaśnić powody ich wyboru, a także przedstawić ich zalety.

Gra jest napisana w całości w języku Lua (http://www.lua.org/). 
Wykorzystuje ona framework LÖVE (https://love2d.org/).
Mapy do gry tworzone były z użyciem edytora map Tiled (http://www.mapeditor.org/).
Użyte zostały trzy biblioteki do LÖVE: Simple Tiled Implementation (STI), 
LÖVE Helper Utilities for Massive Progression (HUMP) oraz TLfres.