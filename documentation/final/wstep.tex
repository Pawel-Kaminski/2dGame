\chapter*{Wstęp}
Niniejsza praca inżynierska składa się z dwóch elementów. Pierwszym elementem
jest gra komputerowa przeznaczona na komputery z zainstalowanym systemem operacyjnym Linux.
Do pracy dołączony jest nośnik zawierający plik wykonywalny o rozszerzeniu
.love, dzięki któremu można uruchomić grę. Drugim elementem jest część pisemna 
dotycząca tej gry.
Rozdział 1 opisuje, w jaki sposób można zainstalować środowisko uruchomieniowe
LÖVE. Po jego instalacji pliki z rozszerzeniem .love będą traktowane przez
system jako pliki wykonywalne. Instalacja LÖVE jest niezbędna do uruchomienia
gry.
Rozdział 2 omawia sterowanie w ramach gry. Czytelnik może dowiedzieć się, w
jaki sposób można sterować postacią, otwierać ekrany zawierające informacje o
ekwipunku, otrzymanych i ukończonych zadaniach lub odblokowanych
umiejętnościach. W tym rozdziale opisano cel gry oraz szczegółowo objaśniono,
w jaki sposób toczyć walki z przeciwnikami.
Rozdział 3 zawiera opis elementów składowych gry. Rozdział ten szczegółowo
opisuje wszystkie elementy, które znalazły się w grze. W tym rozdziale nie jest
poruszany sam problem implementacji, natomiast można go traktować jako opis
efektu końcowego. Przeczytanie tego rozdziału pozwoli czytelnikowi zrozumieć
cele, jakie zostały postawione przed rozpoczęciem procesu implementacji.
Rozdział 4 wymienia i krótko opisuje technologie użyte do osiągnięcia rezultatu
opisanego w poprzednim rozdziale. Zaprezentowane są ogólne informacje o tych
technologiach, jak również powody, które spowodowały wybranie ich do użycia w
projekcie.
Rozdział 5 opisuje mechanizm przełączania pomiędzy stanami gry. Mechanizm ten
jest jednym z kluczowych elementów wykorzystanych w implementacji gry. Rozdział
zapoznaje czytelnika z koncepcją stanów gry, przedstawia zalety ich użycia,
wymienia jakie stany zostały utworzone w procesie implementacji gry oraz opisuje, jakie
funkcje pełni każdy z użytych stanów.
Rozdział 6 stanowi opis logiki gry. W tym rozdziale poruszane są szczegóły
związane z implementacją gry. Czytelnik może zapoznać się z zastosowanym
podziałem na katalogi i pliki. Wyjaśniono tutaj, w jaki sposób zostały osiągnięte cele
określone w rozdziale 3.
Rozdział 7 podsumowuje całość pracy. Pokrótce streszczono najważniejsze
informacje omówione w poprzednich rozdziałach. Czytelnik może przeczytać o tym, jakie
były cele pracy i co udało się osiągnąć. Zawarte są tam również końcowe wnioski
i refleksje, które nasuwają się podczas analizy uzyskanego efektu końcowego,
jakim jest w pełni działająca gra.