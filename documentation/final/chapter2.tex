\chapter{Uruchomienie oraz obsługa gry}
Do uruchomienia gry niezbędna jest instalacja środowiska uruchomieniowego LÖVE.
Poprzedni rozdział opisuje sposób jego instalacji.

Aby uruchomić grę po instalacji LÖVE, wystarczy dwukrotnie kliknąć na plik 
z rozszerzeniem .love.

Po uruchomieniu gry wyświetli się menu główne. Aby rozpocząć nowę grę, należy 
wybrać opcję 'Nowa gra'. Aby nawigować po menu, należy użyć klawiszy 'strzałka 
w górę' oraz 'strzałka w dół'. Aby potwierdzić wybór, należy nacisnąć klawisz 
ENTER.

Następnie gracz ma możliwość wyboru lokacji. Na samym początku dostępna jest 
tylko jedna z czterech lokacji. Należy potwierdzić wybór klawiszem ENTER.
Aby powrócić w przyszłości do tego ekranu, należy odnaleźć teleport na mapie i
potwierdzić powrót klawiszem ENTER.

W dowolnym momencie można opuścić grę, naciskając kombinację klawiszy 'alt' i 
'F4'. Uwaga! Wyjście z gry spowoduje całkowitą utratę postępów osiągniętych 
przez gracza.

Po wybraniu lokacji, gracz jest przenoszony na mapę. Sterowanie na mapie:

klawisze strzałek - poruszaj się po mapie

Q - wyświetl ekran ekwipunku

S - wyświetl listę umiejętności

L - wyświetl listę zadań do wykonania.

Na mapie można spotkać postaci, z którymi można porozmawiać.
Aby porozmawiać z jakąś postacią, należy do niej podejść.
Aby zamknąć okno dialogowe, należy odejść od napotkanej postaci.

Oprócz tego na mapie można znaleźć przydatne lub ważne przedmioty.
Aby wziąć przedmiot do ekwipunku, należy podejść w miejsce, w którym znajduje
się przedmiot.

Na mapie można również spotkać przeciwników.
Przeciwnik znajdujący się na mapie reprezentuje grupkę przeciwników, z którymi 
można stoczyć walkę.
Aby rozpocząć walkę, należy do podejść do wyświetlanego na mapie przeciwnika.
Walki odbywają się na przeznaczonej do tego arenie walki.

Każdy z wojowników charakteryzuje się kilkoma ważnymi parametrami.
Część z nich jest ukryta, a część jest wyświetlana na ekranie.
Parametry, które są wyświetlane na ekranie, to:

HP - punkty życia (health points)

MP - punkty magiczne (magic points)

Obrona - procentowy współczynnik określający odporność na obrażenia

System walki to turowy system walki z elementami czasu rzeczywistego.
Zazwyczaj gracz mierzy się z trójką przeciwników.
Gracz i przeciwnicy mogą wykonywać swoje działania tylko wtedy, gdy trwa ich 
tura.

W przeciwieństwie do klasycznego turowego systemu walki, tury nie muszą
występować na zmianę.
Każda z postaci posiada licznik odliczający czas, który musi upłynąć, aby 
nastąpiła tura tej postaci.
Informacja o czasie pozostałym do wykonania ataku przez przeciwnika jest jawna 
dla gracza. Gdy któryś licznik osiągnie 0, dana postać wykonuje akcję.
Napotkani przeciwnicy mogą być szybsi lub wolniejsi od gracza w swoich 
działaniach, tzn. każdy wojownik posiada swój własny współczynnik czasu 
oczekiwania.

Celem akcji wykonywanej w czasie swojej tury będzie zazwyczaj wykonanie ataku 
odbierającego punkty życia (HP) przeciwnika. Czasami będą dostępne również inne 
rozwiązania, takie jak np. rzucanie czarów defensywnych redukujących przyszłe 
obrażenia. Aby skutecznie walczyć, należy analizować obecną sytuację na polu 
bitwy i wybierać najbardziej opłacalne akcje z listy dostępnych akcji.

Każda postać posiada współczynnik obrony.
Jest on wyrażony w skali od 0 do 100\% i określa, ile procent obrażeń zadawanych 
przez przeciwnika jest unikanych. Ilość punktów odbieranych przez atak jest 
wyrażona wzorem:

efektywna ilość obrażeń = ilość obrażeń, które zadaje atak * (100\% - współczynnik obrony).

Jeżeli liczba punktów życia jakiegoś wojownika spadnie do 0, oznacza to śmierć 
postaci. Jeżeli liczba punktów życia wszystkich przeciwników wyniesie 0, walka 
jest zakończona sukcesem. Gracz jest przenoszony z powrotem na mapę.
Jeżeli liczba punktów życia gracza wyniesie 0, walka jest zakończona porażką.
Gracz otrzymuje wtedy propozycję podjęcia ponownej próby.

W czasie tury wyświetlane są 3 opcje:

Akcja - otwiera listę dostępnych akcji

Przedmiot - otwiera listę przedmiotów do użycia w trakcie walki

Ucieczka - pozwala uciec z ekranu walki na mapę

Podczas nawigowana po menu, można wrócić do poprzedniego menu, naciskając 
klawisz BACKSPACE.

Po wybraniu opcji 'akcja', należy wybrać jedną z akcji z wyświetlonej listy.
Do każdej z akcji wyświetlany jest opis wyjaśniający, co robi dana akcja.
Jeżeli wybrana akcja jest atakiem, należy jeszcze wybrać przeciwnika, którego 
chcemy zaatakować.
Niektóre z akcji wymagają posiadania punktów magicznych (MP).
Jeśli gracz nie posiada wystarczającej liczby MP, należy wybrać inną akcję.

Po wybraniu opcji 'przedmiot', można wybrać jeden z przedmiotów z listy.
W przeciwieństwie do akcji, ich dokładne działanie nie jest znane.
Użycie przedmiotu powoduje jego bezpowrotną utratę.

Opcja 'ucieczka' pozwala zrezygnować z walki. Ta opcja może być przydatna, gdy 
walka z danym przeciwnikiem sprawia dużo problemów. Można wtedy wrócić do mapy 
i poszukać łatwiejszych przeciwników lub zebrać znajdujące się na mapie 
przedmioty.