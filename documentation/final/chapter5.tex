\chapter{Mechanizm przełączania pomiędzy stanami gry}
W rozdziale 4 wspomniano o bibliotece LÖVE Helper Utilities for Massive 
Progression. Biblioteka ta oferuje wiele narzędzi ułatwiających tworzenie gier. 
Jej użycie w projekcie omawianej gry jest związane z bardzo istotnym mechanizmem 
użytym w procesie implementacji gry. Ten mechanizm to stany gry.
Ten rozdział szczegółowo wyjaśni na czym polega ten mechanizm, przedstawi 
zastosowany podział na stany gry oraz uzasadni zastosowanie takiego podziału.

W przedstawianym projekcie utworzono następujące stany gry: menuState, 
mapSelectionState, mapState, battleState oraz questsState.